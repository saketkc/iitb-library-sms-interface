\documentclass[]{beamer}
% Class options include: notes, notesonly, handout, trans,
%                        hidesubsections, shadesubsections,
%                        inrow, blue, red, grey, brown

% Theme for beamer presentation.
\usepackage{amsmath}
\usepackage{cclicenses}
\usepackage{beamerthemeshadow} 
\usepackage{ragged2e} 
\usepackage{lipsum} 
\usepackage{color}
\beamertemplatenavigationsymbolsempty
\beamertemplatenavigationsymbolsempty 

% Other themes include: beamerthemebars, beamerthemelined, 
%                       beamerthemetree, beamerthemetreebars  

\title{AE 663:SDES\\
ASC SMS Interface\\}    % Enter your title between curly braces
\author[Avnish,Giri,Saket,Pushkar]{}
\institute[Aerospace Engineering Department]{
  \begin{tabular} { l c}
  09001007 & Avnish Kumar\\
  09001008 & Giri Prashant\\                                                                                                                   
  09D02007 & Saket Choudhary\\
  09D01005 & Pushkar Godbole \\
  Group & 1\\
   \end{tabular}
  }

\date{\today}

\begin{document}

\begin{frame}
  \titlepage
\end{frame}


\begin{frame}
	\frametitle{Aim of the project}   
    \begin{itemize}
	\justifying
	\item \color{green} Grade notification app \color{black} : which will notify a student about his grade whenever it is updated on ASC
	\item \color{green} Grading statistics app \color{black} : which will sms the grading statistics of a course for a particular year when an sms request is sent
	\item \color{green} Library fine app \color{black} : which will give timely reminders via sms, to the student as the book return due date approaches
	\item \color{green} Moodle update app \color{black} : which will send a notification sms to students on upload of a new lecture in any of the registered courses
	\end{itemize}
\end{frame}

\begin{frame}
	\frametitle{The cake :)}   
  	\begin{itemize}
	\justifying
	\item  The grade notification app sends the present status of the grades on an sms request. Automation was not implemented as admin read-only access to the asc
 website was not available.
	\item The grading stats app works as planned; sending the grading statistics of a course when the course code and year are smsed to it.
	\item Similar to the grade notification app, the library fine app gives the present status of dues and fines on an sms request.
	\item The moodle update app sends an update of all the new uploads in all the registered courses on moodle, upon request.
  	\end{itemize}
\end{frame}

\begin{frame}
	\frametitle{Difficulties Faced}   
  	\begin{itemize}
	\justifying
	\item Lack of admin read-only access to ASC and moodle prevented the automation of smses. Polling 8000 accounts, the only other option was too inefficient to implement.
	\item The interfacing between the python code and the smsing interface (textweb) was not very clearly explained. Understanding the communication between the sites and using their API was time consuming.
	\item Proxy resolution and LDAP authentication posed major challenges.
	\item Being behind proxy, the sms interface could not directly access ASC. Hence re-routing was required for the library and grading stats apps.
This was achieved by placing a php script in homepages.iitb.ac.in. This acts as a node between textweb and ASC.
  	\end{itemize}
\end{frame}

\begin{frame}
	\frametitle{SMS interface (TEXTWEB)}   
  	\begin{itemize}
	\justifying
	\item The aim of this is to get an input from the user and send an output back to him through sms
	\item The google app engine has been used for hosting the code with intefacing with txtweb
	\item This code extracts the information a user sends through the url of the request sent to it

  	\end{itemize}
\end{frame}

\begin{frame}
	\frametitle{Grade notification}   
  	\begin{itemize}
	\justifying
	\item This code will get the grade for a specified semester of a student when given the username and password
	\item The code scrapes the external asc site and extracts the required information 
	\item It uses the modules re,httplib2,beautiful soup. 
	
  	\end{itemize}
\end{frame}

\begin{frame}
	\frametitle{Grading stats}   
  	\begin{itemize}
	\justifying
	\item Function ''gstats'' takes a string like 'AE 663 2011' as arguement and returns a string containing the grading statistics for AE 663 for year 2011 as output.
	\item Packages used : httplib, string
	\item \color{green}asc.iitb.ac.in \color{black} uses get method for every page. This simplified the process of finding a particular grading statistics
	\item The code generates the required url $\rightarrow$ checks whether and in which semester the course was offered $\rightarrow$ scraps the grading page $\rightarrow$ generates the output.
  	\end{itemize}
\end{frame}

\begin{frame}
	\frametitle{Library fine}   
  	\begin{itemize}
	\justifying
	\item Take the LDAP username and password from user through SMS. 
	\item asc.iitb.ac.in has implemented frames. The exact urls to which  the login details are submitted is http://asc.iitb.ac.in/academic/commjsp/ldaplogin.jsp. This made the work alot easier.
	\item The main challenge was scraping! We used httplib2,re for pattern matching and making get and post requests.

  	\end{itemize}
\end{frame}

\begin{frame}
	\frametitle{Moodle update}   
  	\begin{itemize}
	\justifying
	\item The function checks for all recent actvity on a moodle homepage 
	\item packages used: urllib, urllib2, re, cookielib 
	\item given a username-password, it scrapes the webpages for all the courses of that person and return recent activity on the course page. 
  	\end{itemize}
\end{frame}

\begin{frame}
	\frametitle{Data flow}   
  	\begin{itemize}
	\justifying
	\item A php script sits at homepages.iitb.ac.in.
	\item The sms layer code in the google app engine receives a request via textweb
	\item This code sends the request to the php script @ homepages \color{red} $\leftarrow$ (Re-routing)
	\item \color{black} The php script sends it using curl to the localhost where all the python scripts are located
  	\item The corresponding python code is run to generate the required output
	\item This output is sent back to the sms layer code in google app engine
	\item Google app engine sends it back to textweb which converts it to sms and sends it to the requestor
	\end{itemize}
\end{frame}



\end{document}
